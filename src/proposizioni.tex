\chapter{Proposizioni, connettivi e tavole di verità}
\label{ch:aritmetica_modulare}

\section{Proposizioni}
\label{sec:proposizioni}

\begin{definizione}[Proposizione]
    Si chiama \textbf{proposizione} un'espressione (frase) alla quale si può attribuire un valore di verità, cioè
    ha senso domandarsi se è vera o falsa.
\end{definizione}

Esempi:
\begin{itemize}
    \item <<0 è un numero pari>> è una proposizione, perché è una frase vera;
    \item <<6 è un numero primo>> è una proposizione, anche se è una frase falsa;
    \item <<sta piovendo>> è una proposizione, anche se il tuo valore di verità cambia nel tempo;
    \item <<sei alto>> non è una proposizione, perché non c'è un criterio universalmente accettato per capire se una
        persona è alta o meno;
    \item <<Che ore sono?>> non è una proposizione, perché non ha senso domandarsi se una domanda è vera o falsa.
\end{itemize}

\section{Connettivi}
\label{sec:connettivi}

\begin{definizione}[Negazione]
    La \textbf{negazione} è un connettivo che, data una proposizione $P$, ne genera una nuova, indicata con $\neg P$
    che ha valore di verità opposto a $P$.
\end{definizione}

Tavola di verità

Come è resa in italiano

\begin{definizione}[Congiunzione]
    La \textbf{congiunzione} è un connettivo che, date due proposizioni $P$ e $Q$, ne genera una nuova, indicata con
    $P \land Q$ che è vera solo se sia $P$ sia $Q$ sono vere (in tutti gli altri casi è falsa).
\end{definizione}

Tavola di verità

Come è resa in italiano

\begin{definizione}[Disgiunzione]
    La \textbf{disgiunzione} è un connettivo che, date due proposizioni $P$ e $Q$, ne genera una nuova, indicata con
    $P \lor Q$ che è falsa solo se sia $P$ sia $Q$ sono false (in tutti gli altri casi è vera).
\end{definizione}

Tavola di verità

Come è resa in italiano

Disgiunzione inclusiva vs disgiunzione esclusiva

\begin{definizione}[Implicazione]
    L'\textbf{implicazione} è un connettivo che, date due proposizioni $P$ e $Q$, ne genera una nuova, indicata con
    $P \rightarrow Q$ che è falsa solo se $P$ è falsa e $Q$ è vera (in tutti gli altri casi è vera).

    Le proposizioni $P$ e $Q$ sono chiamate rispettivamente \textbf{antecedente} e \textbf{conseguente},
    oppure \emph{premessa} e \emph{conclusione}, oppure \emph{ipotesi} e \emph{tesi}.
\end{definizione}

Tavola di verità

Come è resa in italiano

Causa/effetto

Comportamento quando P è falsa.

\begin{definizione}[Doppia implicazione]
    La \textbf{doppia implicazione} è un connettivo che, date due proposizioni $P$ e $Q$, ne genera una nuova, indicata con
    $P \leftrightarrow Q$ che è versa solo se $P$ e $Q$ hanno lo stesso valore di verità (in tutti gli altri casi è falsa).
\end{definizione}

Tavola di verità

Come è resa in italiano

\section{Tavole di verità}
\label{sec:tavole_di_verita}

\section{L'antinomia di Russell}
\label{sec:antinomia_di_russell}