\chapter{Esercizi e quiz}
\label{ch:quiz}

%\section{Dall'eserciziario dello stage senior}
%\label{sec:quiz_stage_senior}
%
%\begin{esercizio}
%    \label{ex:stage_senior_41}
%\end{esercizio}

\section{Quiz dai giochi di Archimede}
\label{sec:quiz_giochi_archimede}

\begin{esercizio}[Giochi di Archimede 1996 - Biennio]
    \label{ex:archimede_1996_biennio_13}
    Cinque persone non si trovano d’accordo sulla data.

    – Carlo dice che oggi è lunedì 16 agosto

    – Franco dice che oggi è martedì 16 agosto

    – Marco dice che oggi è martedì 17 settembre

    – Roberto dice che oggi è lunedì 17 agosto

    – Tullio dice che oggi è lunedì 17 settembre.

    Uno ha ragione, ma nessuno ha “completamente” torto, nel senso che ciascuno
    dice correttamente almeno una cosa (o il giorno della settimana, o il giorno del
    mese, o il mese).
    Chi ha ragione?

    (A) Carlo \quad (B) Franco \quad (C) Marco \quad (D) Roberto \quad (E) Tullio.
\end{esercizio}

\begin{esercizio}[Giochi di Archimede 1998 - Biennio]
    \label{ex:archimede_1998_biennio_17}
    Su un’isola vivono tre categorie di persone: i cavalieri, che dicono sempre la verità, i furfanti,
    che mentono sempre, ed i paggi che dopo una verità dicono sempre una menzogna e viceversa.
    Sull’isola incontro un vecchio, un ragazzo e una ragazza.
    Il vecchio afferma: “Io sono paggio”;
    “Il ragazzo è cavaliere”.
    Il ragazzo dice: “Io sono cavaliere”;
    “La ragazza è paggio”.
    La ragazza afferma infine: “Io sono furfante”;
    “Il vecchio è paggio”.
    Si può allora affermare che tra i tre:

    (A) c’è esattamente un paggio

    (B) ci sono esattamente due paggi

    (C) ci sono esattamente tre paggi

    (D) non c’è alcun paggio

    (E) il numero dei paggi non è sicuro.
\end{esercizio}

\begin{esercizio}[Giochi di Archimede 1999 - Triennio]
    \label{ex:archimede_1999_triennio_238}
    Nell’isola dei cavalieri e dei furfanti i cavalieri dicono sempre la verità ed i furfanti mentono sempre.
    Supponi di incontrarvi Andrea che dice “Bruno afferma che Carlo è un furfante,
    ma Carlo afferma che Diego è un furfante e Diego afferma che Bruno è un furfante”.
    Che cosa puoi dedurne?

    (A) Bruno, Carlo e Diego sono tutti furfanti

    (B) Bruno, Carlo e Diego sono tutti cavalieri

    (C) tra Bruno, Carlo e Diego ci sono due furfanti e un cavaliere

    (D) tra Bruno, Carlo e Diego ci sono due cavalieri e un furfante

    (E) Andrea è un furfante.
\end{esercizio}

\begin{esercizio}[Giochi di Archimede 2000 - Biennio]
    \label{ex:archimede_2000_biennio_18}

    Anna, Barbara, Chiara e Donatella si sono sfidate in una gara di nuoto fino alla boa.
    All’arrivo non ci sono stati ex aequo.
    Al ritorno, Anna dice: “Chiara è arrivata prima di Barbara”;
    Barbara dice: “Chiara è arrivata prima di Anna”;
    Chiara dice: “Io sono arrivata seconda”.
    Sapendo che una sola di esse ha detto la verità,

    (A) si può dire solo chi ha vinto

    (B) si può dire solo chi è arrivata seconda

    (C) si può dire solo chi è arrivata terza

    (D) si può dire solo chi è arrivata ultima

    (E) non si può stabilire la posizione in classifica di nessuna.
\end{esercizio}

\begin{esercizio}[Giochi di Archimede 2001 - Biennio]
    \label{ex:archimede_2001_biennio_10}
    L’impiegato del censimento nell’isola dei Cavalieri e Furfanti deve determinare il tipo (Cavalieri o Furfanti) e il
    titolo di studio degli abitanti (i Furfanti mentono sempre, mentre i Cavalieri dicono sempre la verità).
    In un appartamento abitato da due coniugi ottiene solo queste risposte:

    Marito: \emph{siamo entrambi laureati.}

    Moglie: \emph{siamo entrambi furfanti.}

    Quante caselle può riempire con sicurezza l’impiegato?

    (A) 0 \quad (B) 1 \quad (C) 2 \quad (D) 3 \quad (E) 4.
\end{esercizio}

\begin{esercizio}[Giochi di Archimede 2003 - Biennio]
    \label{ex:archimede_2003_biennio_14}

    \framebox[\textwidth]{%
    \begin{minipage}{\textwidth}
    \centering
    In questo rettangolo c'è esattamente una affermazione falsa.

    In questo rettangolo ci sono esattamente due affermazioni false.

    In questo rettangolo ci sono esattamente tre affermazioni false.

    In questo rettangolo ci sono esattamente quattro affermazioni false.
    \end{minipage}}

    Quante affermazioni vere ci sono nel rettangolo?

    (A) 0 \quad (B) 1 \quad (C) 2 \quad (D) 3 \quad (E) 4.
\end{esercizio}

\begin{esercizio}[Giochi di Archimede 2005 - Biennio]
    \label{ex:archimede_2005_biennio_6}
    Una stanza rettangolare ha le pareti rivolte nelle direzioni dei quattro punti cardinali e ci sono quattro porte
    d’accesso.
    Tre persone si trovano nella stanza e fanno le seguenti affermazioni.
    Prima persona: “Non ci sono porte sulla parete Sud”.
    Seconda persona: “Ci sono porte solo sulla parete Nord”.
    Terza persona: “Su ogni parete c’è al massimo una porta”.
    Che cosa si può dire per certo delle affermazioni fatte?

    (A) L’affermazione fatta dalla prima persona è vera

    (B) l’affermazione fatta dalla seconda persona è vera

    (C) l’affermazione fatta dalla terza persona è vera

    (D) almeno una affermazione è falsa

    (E) non si può dire niente di certo sulle affermazioni fatte.
\end{esercizio}

\begin{esercizio}[Giochi di Archimede 2005 - Triennio]
    \label{ex:archimede_2005_triennio_21}
    Quattro bambine, Alice, Bianca, Cecilia e Daniela, decidono di comprare un palloncino a testa da un venditore che
    ha solo palloncini rossi e blu.
    Compreranno il palloncino una dopo l’altra: prima Alice, poi Bianca, poi Cecilia e infine Daniela.
    Bianca dice: “Se Alice lo comprerà rosso, anch’io lo comprerò rosso”.
    Cecilia dice: “Io lo comprerò dello stesso colore di Bianca”.
    Daniela dice: “Se Alice lo comprerà blu, io lo comprerò dello stesso colore di Cecilia”.
    Quale delle seguenti affermazioni è sicuramente vera?

    (A) \'{E} impossibile che quattro bambine comprino un palloncino rosso.

    (B) Almeno tre bambine compreranno un palloncino dello stesso colore.

    (C) Daniela e Bianca compreranno un palloncino dello stesso colore.

    (D) Almeno due bambine compreranno un palloncino rosso.

    (E) Nessuna delle precedenti affermazioni è sicuramente vera
\end{esercizio}

\begin{esercizio}[Giochi di Archimede 2006 - Biennio]
    \label{ex:archimede_2006_biennio_10}
    In un sacchetto ci sono alcune biglie.
    Maria dice: “Nel sacchetto ci sono in tutto tre biglie e sono nere”.
    Luca dice: “Nel sacchetto ci sono due biglie nere e due biglie rosse”.
    Giorgio dice: “Nel sacchetto ci sono solo biglie nere”.
    Sapendo che uno solo dei tre ha mentito, quante biglie ci sono nel sacchetto?

    (A) una \quad (B) due \quad (C) tre \quad (D) quattro \quad
    (E) non è possibile determinarne il numero in base ai dati del problema.
\end{esercizio}

\begin{esercizio}[Giochi di Archimede 2006 - Triennio]
    \label{ex:archimede_2006_triennio_19}
    Gli abitanti di un’isola si dividono in due categorie: quelli che sono sempre sinceri e quelli che mentono sempre.
    Fra tre abitanti dell’isola, Andrea, Barbara e Ciro, avviene questa conversazione:
    Andrea dice: “Barbara `e sincera”, Barbara dice: “Andrea e Ciro sono sinceri”, Ciro dice: “Andrea è bugiardo”.
    Possiamo concludere che

    (A) sono tutti e tre sinceri,

    (B) sono tutti e tre bugiardi,

    (C) Andrea e Barbara sono sinceri e Ciro è bugiardo,

    (D) Andrea e Barbara sono bugiardi e Ciro è sincero,

    (E) Andrea è sincero e Ciro e Barbara sono bugiardi.
\end{esercizio}

\begin{esercizio}[Giochi di Archimede 2007 - Biennio]
    \label{ex:archimede_2007_biennio_19}
    In un paese abitano solo briganti, che mentono sempre, e cavalieri, che dicono sempre la verità.
    Un giornalista intervista quattro abitanti: Arturo, Bernardo, Carlo e Dario, che fanno le seguenti dichiarazioni.
    Arturo: “Bernardo è un brigante”;
    Bernardo: “Io sono l’unico cavaliere tra noi quattro”;
    Carlo: “Almeno uno tra Arturo e Dario è un brigante”;
    Dario: “Siamo 4 cavalieri”.
    Quanti tra i quattro sono cavalieri?

    (A) Nessuno, \quad (B) uno, \quad (C) due, \quad (D) tre, \quad (E) quattro.
\end{esercizio}

\begin{esercizio}[Giochi di Archimede 2007 - Triennio]
    \label{ex:archimede_2007_triennio_22}
    Dopo la scuola Francesco invita i suoi amici a casa sua per studiare e fare merenda e dice:
    “Se saremo in pochi studieremo bene;
    se saremo in tanti mangeremo poco”.
    Quale delle seguenti affermazioni è certamente vera secondo Francesco?

    (A) Se si è in pochi si mangia molto,

    (B) per studiare bene è necessario essere in pochi,

    (C) se si studia male non si è in pochi,

    (D) se si mangia poco si è necessariamente in tanti,

    (E) se si è in tanti si studia male.
\end{esercizio}

\begin{esercizio}[Giochi di Archimede 2008 - Triennio]
    \label{ex:archimede_2008_triennio_21}
    Ogni volta che Agilulfo torna a casa da scuola dopo aver preso un brutto voto,
    se la sua mamma è in casa lo mette in punizione.
    Sapendo che ieri pomeriggio Agilulfo non è stato messo in punizione,
    quale delle seguenti affermazioni è certamente vera?

    (A) Ieri Agilulfo ha preso un brutto voto,

    (B) ieri Agilulfo non ha preso un brutto voto,

    (C) ieri pomeriggio la sua mamma era in casa,

    (D) ieri pomeriggio la sua mamma non era in casa,

    (E) nessuna delle precedenti affermazioni è certamente vera.
\end{esercizio}

%20/2009B
\begin{esercizio}[Giochi di Archimede 2009 - Biennio]
    \label{ex:archimede_2009_biennio_20}
    Quattro amici, Anna, Bea, Caio e Dino, giocano a poker con 20 carte di uno stesso
    mazzo: i quattro re, le quattro regine, i quattro fanti, i quattro assi e i quattro dieci.
    Vengono distribuite cinque carte a testa.
    Anna dice: “Io ho un poker!”
    Bea dice: “Io ho tutte e cinque le carte di cuori”.
    Caio dice: “Io ho cinque carte rosse”.
    Infine Dino dice: “Io ho tre carte di uno stesso
    valore e anche le altre due hanno tra loro lo stesso valore”.
    Sappiamo che una e una sola delle affermazioni è falsa;
    chi sta mentendo?

    (A) Anna, \quad (B) Bea, \quad (C) Caio, \quad (D) Dino, \quad (E) non è possibile determinarlo.
\end{esercizio}

%11/2009T
\begin{esercizio}[Giochi di Archimede 2009 - Triennio]
    \label{ex:archimede_2009_triennio_11}
    La faccia nascosta della luna è popolata solo da furfanti, che mentono sempre,
    cavalieri che dicono sempre il vero, e paggi, che quando pronunciano due frasi consecutive,
    mentono su una e dicono il vero nell’altra, scegliendo in modo casuale
    l’ordine tra le due.
    Tre abitanti, Drago, Ludovico e Orlando, fanno le seguenti
    affermazioni.
    Drago: “Io sono un paggio.
    Ludovico è un cavaliere”.
    Ludovico: “Orlando `e un paggio.
    Io sono un furfante”.
    Orlando: “Io sono un paggio.
    Siamo tutti paggi!”.
    Quanti di loro sono effettivamente paggi?

    (A) 0, \quad (B) 1, \quad (C) 2, \quad (D) 3, \quad (E) non si pu`{o} determinare con i dati a
    disposizione.
\end{esercizio}

%18/2010B
\begin{esercizio}[Giochi di Archimede 2010 - Biennio]
    \label{ex:archimede_2010_biennio_18}
    Un celebre investigatore sta cercando il colpevole di un omicidio tra cinque sospettati: Anna, Bruno, Cecilia,
    Dario ed Enrico.
    Egli sa che il colpevole mente sempre e gli altri dicono sempre la verit`{a}.
    Anna afferma: “Il colpevole è un maschio!”.
    Cecilia dice: “`E stata Anna oppure è stato Enrico”.
    Infine Enrico dice: “Se Bruno è colpevole allora Anna è innocente”.
    Chi ha commesso l’omicidio?

    (A) Anna, \quad (B) Bruno, \quad (C) Cecilia, \quad (D) Dario, \quad (E) Enrico.
\end{esercizio}

%19/2010T
\begin{esercizio}[Giochi di Archimede 2010 - Triennio]
    \label{ex:archimede_2010_triennio_19}
    Il maggiore Tom è atterrato su un pianeta popolato da gatti viola, che dicono
    sempre la verit`{a}, e da gatti neri, che mentono sempre.
    Nel buio pi`{u} completo incontra 5 gatti, che si rivolgono a lui nel modo seguente.
    Primo gatto: “Sono viola”;
    secondo gatto: “Almeno 3 di noi sono viola”;
    terzo gatto: “Il primo gatto è nero”;
    quarto gatto: “Almeno 3 di noi sono neri”;
    quinto gatto: “Siamo tutti neri”.
    Quanti dei 5 gatti sono viola?

    (A) nessuno, \quad (B) 1, \quad (C) 2, \quad (D) 3, \quad (E) 4.
\end{esercizio}

%13/2011B
\begin{esercizio}[Giochi di Archimede 2011 - Biennio]
    \label{ex:archimede_2011_biennio_13}
    Dopo una rissa in campo l’arbitro vuole espellere il capitano di una squadra di calcio.
    `{E} uno tra Paolo, Andrea e Gabriele ma, siccome nessuno ha la fascia al braccio, non sa qual è dei tre.
    Paolo dice di non essere il capitano;
    Andrea dice che il capitano è Gabriele;
    Gabriele dice che il capitano è uno degli altri due.
    Sapendo che uno solo dei tre dice la verit`{a}, quale delle affermazioni seguenti è sicuramente vera?

    (A) Gabriele non è il capitano,

    (B) Andrea dice la verit`{a},

    (C) Paolo dice la verit`{a},

    (D) Andrea è il capitano,

    (E) Gabriele mente.
\end{esercizio}

%23/2011T
\begin{esercizio}[Giochi di Archimede 2011 - Triennio]
    \label{ex:archimede_2011_triennio_23}
    La polizia indaga su una rapina.
    I cinque indagati, tra cui c’è sicuramente il colpevole e forse anche qualche suo complice, interrogati
    dichiarano:
    A: “B è colpevole.
    D è uno dei complici.”
    B: “E è innocente.
    A è uno dei complici.”
    C: “E è il colpevole.
    D è innocente.”
    D: “Il colpevole è effettivamente E.
    A è stato suo complice.”
    E: “A era uno dei complici.
    C è il colpevole.”
    Sapendo che il colpevole mente su tutto, gli eventuali complici, per paura, rendono una dichiarazione vera ed una
    falsa e le persone innocenti, infine, dicono sempre la verit`{a}, quanti sono i complici?

    (A) 0, \quad (B) 1, \quad (C) 2, \quad (D) 3, \quad (E) è impossibile determinarlo.
\end{esercizio}

%15/2011T
\begin{esercizio}[Giochi di Archimede 2011 - Triennio]
    \label{ex:archimede_2011_triennio_15}
    Ciascuno dei quattro amici Anna, Erica, Lorenzo e Giuseppe, mente sempre o dice sempre la verit`{a}.
    Anna dice: “Erica mente sempre”;
    Erica dice: “Giuseppe dice sempre il vero”;
    Giuseppe dice: “Anna mente sempre”;
    infine Lorenzo dice: “Anna, Erica e Giuseppe mentono sempre”.
    Quanti sono, al massimo, quelli che mentono sempre?
    (A) 1, \quad (B) 2, \quad (C) 3, \quad (D) 4, \quad (E) nessuna delle precedenti.
\end{esercizio}

%12/2015B
\begin{esercizio}[Giochi di Archimede 2015 - Biennio]
    \label{ex:archimede_2015_biennio_12}
    Sull'isola dei cavalieri e dei furfanti, i cavalieri sono sempre sinceri ed i furfanti mentono sempre.
    Durante una riunione, i presenti si siedono attorno a un grande tavolo e ciascuno dice:
    “la persona alla mia destra è un furfante”.
    Sapendo che tra i presenti ci sono meno di 100 cavalieri, quale dei seguenti potrebbe essere il numero dei
    partecipanti alla riunione?

    (A) 208 \quad (B) 85 \quad (C) 153 \quad (D) 168 \quad (E) 205
\end{esercizio}

%3/2016B
\begin{esercizio}[Giochi di Archimede 2016 - Biennio]
    \label{ex:archimede_2016_biennio_3}
    In un'isola vivono due tipi di persone: i cavalieri che dicono sempre la verità ed i furfanti che mentono sempre.
    Durante una festa di compleanno, alla quale partecipano 450 persone, ciascuno dei presenti afferma:
    “tutti coloro che, oltre a me, sono presenti alla festa sono dei furfanti”.
    Quanti sono i furfanti alla festa?

    (A) nessuno \quad (B) 450 \quad (C) 449 \quad (D) 224 \quad (E) 225
\end{esercizio}

%2/2017B
\begin{esercizio}[Giochi di Archimede 2017 - Biennio]
    \label{ex:archimede_2017_biennio_2}
    Attorno a un tavolo sono sedute 4 persone, ciascuna delle quali può essere o un cavaliere (che dice sempre la
    verità) o un furfante (che mente sempre).
    Ognuno dei presenti afferma:
    “Delle altre tre persone sedute a questo tavolo insieme a me, i furfanti sono esattamente due”.
    Qual è il numero complessivo di furfanti che sono seduti al tavolo?

    (A) nessuno

    (B) sicuramente 2

    (C) sicuramente tutti e 4

    (D) sicuramente 1

    (E) gli elementi forniti non sono sufficienti per stabilirlo
\end{esercizio}

%4/2017T
\begin{esercizio}[Giochi di Archimede 2017 - Triennio]
    \label{ex:archimede_2017_triennio_4}
    Attorno a un tavolo circolare sono sedute 6 persone, ciascuna delle quali può
    essere o un cavaliere (che dice sempre la verità) o un furfante (che mente sempre).
    Ognuno dei presenti afferma: “Considerando i miei due vicini e la persona che
    è seduta proprio di fronte a me, esattamente due di queste tre persone sono
    furfanti”.
    Quanti sono, in tutto, i cavalieri seduti al tavolo?

    (A) nessuno \quad (B) 1 \quad (C) 3 \quad (D) 4 \quad
    (E) gli elementi forniti non sono sufficienti per stabilirlo
\end{esercizio}

%15/2019B
\begin{esercizio}[Giochi di Archimede 2019 - Biennio]
    \label{ex:archimede_2019_biennio_15}
    Attorno a un tavolo ci sono 8 persone, ciascuna delle quali può essere o un cavaliere
    o un furfante.
    Ogni volta che parla un cavaliere, la frase che pronuncia è vera;
    ogni volta che parla un furfante, la frase che pronuncia è falsa.
    Uno di loro pronuncia la seguente frase:
    “alla mia destra siede un cavaliere e alla mia sinistra siede un furfante”.
    Il vicino di destra di costui dichiara: “alla mia sinistra siede un cavaliere e alla mia destra siede un furfante”.
    Il vicino di destra di quest'ultimo afferma:
    “alla mia destra siede un cavaliere e alla mia sinistra siede un furfante”.
    E così via, le frasi si alternano, fino all'ottava persona, che afferma:
    “alla mia sinistra siede un cavaliere e alla mia destra siede un furfante”.
    Si può concludere che, tra le 8 persone presenti, il numero complessivo di cavalieri \dots

    (A) è possibile che sia 2 oppure 4, ma non 0, 6 o 8

    (B) è possibile che sia 2, 4 o 6, ma non 0 o 8

    (C) è possibile che sia 0, 2 o 4, ma non 6 o 8

    (D) è possibile che sia 0, 2, 4 o anche 6, ma non 8

    (E) è possibile che sia 0 oppure 4, ma non 2, 6 o 8
\end{esercizio}

%\section{Quiz dalle gare distrettuali}
%\label{sec:quiz_gare_distrettuali}

\section{Quiz da Kangourou della Matematica, categoria Cadet}
\label{sec:quiz_kangourou_cadet}

\begin{esercizio}[Cadet 2010]
    \label{ex:cadet_2010}
    In una città vivono solo gentiluomini e bugiardi.
    Ogni singola affermazione pronunciata da un gentiluomo è vera,
    mentre ogni singola affermazione pronunciata da un bugiardo è falsa.
    Alcuni abitanti sono riuniti in una stanza e tre di loro fanno, una per ciascuno, le seguenti coppie di affermazioni:
    \begin{enumerate}
        \item <<Non ci sono più di tre persone in questa stanza>> e <<Tutti noi mentiamo>>;
        \item <<Non ci sono più di quattro persone in questa stanza>> e <<Non siamo tutti bugiardi>>;
        \item <<Ci sono cinque persone in questa stanza>> e <<Tre di noi sono bugiardi>>.
    \end{enumerate}
    Quante persone ci sono nella stanza e quanti bugiardi ci sono tra di loro?

    A) 3 persone, 1 bugiardo

    B) 4 persone, 1 bugiardo

    C) 4 persone, 2 bugiardi

    D) 5 persone, 2 bugiardi

    E) 5 persone, 3 bugiardi
\end{esercizio}

\begin{esercizio}[Cadet 2014]
    \label{ex:cadet_2014}
    In una stanza ci sono 25 persone ciascuna delle quali appartiene a una e una sola delle seguenti confraternite:
    i Verdi che dicono sempre la verità, i Rossi che mentono sempre, e i gialli che, se a una domanda rispondono in vero,
    alla successiva mentono, e viceversa.
    A ciascuna di queste persone è stato chiesto nell'ordine:
    \begin{itemize}
        \item <<Sei dei Verdi?>> e 17 hanno risposto sì,
        \item <<Sei dei Gialli?>> e 12 hanno risposto sì,
        \item <<Sei dei Rossi?>> e 8 hanno risposto sì.
    \end{itemize}
    Quanti sono gli appartenenti ai Verdi?

    A) 4 \quad B) 5 \quad C) 9 \quad D) 13 \quad E) 17
\end{esercizio}

\section{Altri quiz}
\label{sec:quiz_altri}

\begin{esercizio}[Dal libro P. Fiorini, “Manuale di allenamento per le gare di matematica”, Scienza Express edizioni]
    \label{ex:fiorini_1_1_4}
    Sull'isola del tesoro ci sono tre naufraghi, che chiameremo Arthur, Martin e Raymond.
    Chiediamo ad Arthur se è un cavaliere o un furfante, ma non riusciamo a sentire la sua risposta.
    Martin allora dice: “Arthur ha detto di essere un furfante” e Raymond dice “Non credere a Martin: sta mentendo!”.
    Quanti sono i furfanti?
\end{esercizio}
